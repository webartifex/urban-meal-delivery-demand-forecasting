\subsubsection{Horizontal and Whole-day-ahead Forecasts.}
\label{hori}

The upper-left in Figure \ref{f:inputs} illustrates the simplest way to
    generate forecasts for a test day before it has started:
For each time of the day, the corresponding horizontal slice becomes the input
    for a model.
With whole days being the unified time interval, each model is trained $H$
    times, providing a one-step-ahead forecast.
While it is possible to have models of a different type be selected per time
    step, that did not improve the accuracy in the empirical study.
As the models in this family do not include the test day's demand data in
    their training sets, we see them as benchmarks to answer \textbf{Q4},
    checking if a UDP can take advantage of real-time information.
The models in this family are as follows; we use prefixes, such as "h" here,
    when methods are applied in other families as well:
\begin{enumerate}
\item \textit{\gls{naive}}:
          Observation from the same time step one week prior
\item \textit{\gls{trivial}}:
          Predict $0$ for all time steps
\item \textit{\gls{hcroston}}:
          Intermittent demand method introduced by \cite{croston1972}
\item \textit{\gls{hholt}},
      \textit{\gls{hhwinters}},
      \textit{\gls{hses}},
      \textit{\gls{hsma}}, and
      \textit{\gls{htheta}}:
          Exponential smoothing without calibration
\item \textit{\gls{hets}}:
          ETS calibrated as described by \cite{hyndman2008b}
\item \textit{\gls{harima}}:
          ARIMA calibrated as described by \cite{hyndman2008a}
\end{enumerate}
\textit{naive} and \textit{trivial} provide an absolute benchmark for the
    actual forecasting methods.
\textit{hcroston} is often mentioned in the context of intermittent demand;
    however, the method did not perform well at all.
Besides \textit{hhwinters} that always fits a seasonal component, the
    calibration heuristics behind \textit{hets} and \textit{harima} may do so
    as well.
With $k=7$, an STL decomposition is unnecessary here.
