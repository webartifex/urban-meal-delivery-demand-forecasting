\subsubsection{Vertical and Whole-day-ahead Forecasts without Retraining.}
\label{vert}

The upper-right in Figure \ref{f:inputs} shows an alternative way to
    generate forecasts for a test day before it has started:
First, a seasonally-adjusted time series $a_t$ is obtained from a vertical
    time series by STL decomposition.
Then, the actual forecasting model, trained on $a_t$, makes an $H$-step-ahead
    prediction.
Lastly, we add the $H$ seasonal na\"{i}ve forecasts for the seasonal component
    $s_t$ to them to obtain the actual predictions for the test day.
Thus, only one training is required per model type, and no real-time data is
    used.
By decomposing the raw time series, all long-term patterns are assumed to be
    in the seasonal component $s_t$, and $a_t$ only contains the level with
    a potential trend and auto-correlations.
The models in this family are:
\begin{enumerate}
\item \textit{\gls{fnaive}},
      \textit{\gls{pnaive}}:
          Sum of STL's trend and seasonal components' na\"{i}ve forecasts
\item \textit{\gls{vholt}},
      \textit{\gls{vses}}, and
      \textit{\gls{vtheta}}:
          Exponential smoothing without calibration and seasonal
                       fit
\item \textit{\gls{vets}}:
          ETS calibrated as described by \cite{hyndman2008b}
\item \textit{\gls{varima}}:
          ARIMA calibrated as described by \cite{hyndman2008a}
\end{enumerate}
As mentioned in Sub-section \ref{unified_cv}, we include the sum of the
    (seasonal) na\"{i}ve forecasts of the STL's trend and seasonal components
    as forecasts on their own:
For \textit{fnaive}, we tune the "flexible" $ns$ parameter, and for
    \textit{pnaive}, we set it to a "periodic" value.
Thus, we implicitly assume that there is no signal in the remainder $r_t$, and
    predict $0$ for it.
\textit{fnaive} and \textit{pnaive} are two more simple benchmarks.
