\subsection{Overall Approach}
\label{approach_approach}

On a conceptual level, there are three distinct aspects of the model
    development process.
First, a pre-processing step transforms the platform's tabular order data into
    either time series in Sub-section \ref{grid} or feature matrices in
    Sub-section \ref{ml_models}.
Second, a benchmark methodology is developed in Sub-section \ref{unified_cv}
    that compares all models on the same scale, in particular, classical
    models with ML ones.
Concretely, the CV approach is adapted to the peculiar requirements of
    sub-daily and ordinal time series data.
This is done to maximize the predictive power of all models into the future
    and to compare them on the same scale.
Third, the forecasting models are described with respect to their assumptions
    and training requirements.
Four classification dimensions are introduced:
\begin{enumerate}
\item \textbf{Timeliness of the Information}:
    whole-day-ahead vs. real-time forecasts
\item \textbf{Time Series Decomposition}: raw vs. decomposed
\item \textbf{Algorithm Type}: "classical" statistics vs. ML
\item \textbf{Data Sources}: pure vs. enhanced (i.e., with external data)
\end{enumerate}
Not all of the possible eight combinations are implemented; instead, the
    models are varied along these dimensions to show different effects and
    answer the research questions.
