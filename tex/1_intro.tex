\section{Introduction}
\label{intro}

In recent years, many meal delivery platform providers (e.g., Uber Eats,
    GrubHub, DoorDash, Deliveroo) with different kinds of business models have
    entered the markets in cities around the world.
A study by \cite{hirschberg2016} estimates the global market size to surpass
    20 billion Dollars by 2025.
A common feature of these platforms is that they do not operate kitchens but
    focus on marketing their partner restaurants' meals, unifying all order
    related processes in simple smartphone apps, and managing the delivery via
    a fleet of either employees or crowd-sourced sub-contractors.

Various kinds of urban delivery platforms
    (\gls{udp}; \ref{glossary} provides a glossary with all abbreviations)
    have received attention in recent scholarly publications.
\cite{hou2018} look into heuristics to simultaneously optimize courier
    scheduling and routing in general, while \cite{masmoudi2018} do so
    for the popular dial-a-ride problem and \cite{wang2018} investigate
    the effect of different fulfillment strategies in the context of urban
    meal delivery.
\cite{ehmke2018} and \cite{alcaraz2019} focus their research on the routing
    aspect, which is commonly modeled as a so-called vehicle routing problem
    (\gls{vrp}).

Not covered in the recent literature is research focusing on the demand
    forecasting problem a UDP faces.
Due to the customers' fragmented locations and the majority of the orders
    occurring ad-hoc for immediate delivery in the case of a meal delivery
    platform, forecasting demand for the near future (i.e., several hours)
    and distinct locations of the city in real-time is an essential factor
    in achieving timely fulfillment.
In general, demand forecasting is a well-researched discipline with a
    decades-long history in scholarly journals as summarized, for example, by
    \cite{de2006}.
Even some meal delivery platforms themselves publish their practices: For
    example, \cite{bell2018} provide a general overview of supply and demand
    forecasting at Uber and benchmarks of the methods used while
    \cite{laptev2017} investigate how extreme events can be incorporated.

The conditions such platforms face are not limited to meal delivery:
    Any entity that performs ad-hoc requested point-to-point transportation at
    scale in an urban area benefits from a robust forecasting system.
Examples include ride-hailing, such as the original Uber offering, or bicycle
    courier services.
The common characteristics are:
\begin{itemize}
\item \textbf{Geospatial Slicing}:
    Forecasts for distinct parts of a city in parallel
\item \textbf{Temporal Slicing}:
    Forecasts on a sub-daily basis (e.g., 60-minute windows)
\item \textbf{Order Sparsity}:
    The historical order time series exhibit an intermittent pattern
\item \textbf{Double Seasonality}:
    Demand varies with the day of the week and the time of day
\end{itemize}
Whereas the first two points can be assumed to vary with the concrete
	application's requirements, it is the last two that pose challenges for
	forecasting a platform's demand:
Intermittent demand (i.e., many observations in the historic order time series
    exhibit no demand at all) renders most of the commonly applied error
    metrics useless.
Moreover, many of the established forecasting methods can only handle a single
    and often low seasonality (i.e., repeated regular pattern), if at all.

In this paper, we develop a rigorous methodology as to how to build and
    evaluate a robust forecasting system for an UDP
    that offers ad-hoc point-to-point transportation of any kind.
We implement such a system with a broad set of commonly used forecasting
    methods.
We not only apply established (i.e., "classical") time series methods but also
    machine learning (\gls{ml}) models that have gained traction in recent
    years due to advancements in computing power and availability of larger
    amounts of data.
In that regard, the classical methods serve as benchmarks for the ML methods.
Our system is trained on and evaluated with a dataset obtained from an
    undisclosed industry partner that, during the timeframe of our study, was
    active in several European countries and, in particular, in France.
Its primary business strategy is the delivery of meals from upper-class
    restaurants to customers in their home or work places via bicycles.
In this empirical study, we identify the best-performing methods.
Thus, we answer the following research questions:
\begin{enumerate}
\item[\textbf{Q1}:]
    Which forecasting methods work best under what circumstances?
\item[\textbf{Q2}:]
    How do classical forecasting methods compare with ML models?
\item[\textbf{Q3}:]
    How does the forecast accuracy change with more historic data available?
\item[\textbf{Q4}:]
    Can real-time information on demand be exploited?
\item[\textbf{Q5}:]
    Can external data (e.g., weather data) improve the forecast accuracy?
\end{enumerate}
To the best of our knowledge, no such study has yet been published in a
    scholarly journal.

The subsequent Section \ref{lit} reviews the literature on the forecasting
    methods included in the system.
Section \ref{mod} introduces our forecasting system, and Section \ref{stu}
    discusses the results obtained in the empirical study.
Lastly, Section \ref{con} summarizes our findings and concludes
    with an outlook on further research opportunities.