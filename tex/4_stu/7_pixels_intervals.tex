\subsection{Effects of the Pixel Size and Time Step Length}
\label{pixels_intervals}

As elaborated in Sub-section \ref{grid}, more order aggregation leads to a
    higher overall demand level and an improved pattern recognition in the
    generated time series.
Consequently, individual cases tend to move to the right in tables equivalent
    to Table \ref{t:results}.
With the same $ADD$ clusters, forecasts for pixel sizes of $2~\text{km}^2$ and
    $4~\text{km}^2$ or time intervals of 90 and 120 minutes or combinations
    thereof yield results similar to the best models as revealed in Tables
    \ref{t:results}, \ref{t:hori}, \ref{t:vert}, and \ref{t:ml} for high
    demand.
By contrast, forecasts for $0.5~\text{km}^2$ pixels have most of the cases
    (i.e., $n$) in the no or low demand clusters.
In that case, the pixels are too small, and pattern recognition becomes
    harder.
While it is true, that \textit{trivial} exhibits the overall lowest MASE
    for no demand cases, these forecasts become effectively worthless for
    operations.
In the extreme, with even smaller pixels we would be forecasting $0$ orders
    in all pixels for all time steps.
In summary, the best model and its accuracy are determined primarily by the
    $ADD$, and the pixel size and interval length are merely parameters to
    control that.
The forecaster's goal is to create a grid with small enough pixels without
    losing a recognizable pattern.
