\subsection{Overall Results}
\label{overall_results}

Table \ref{t:results} summarizes the overall best-performing models grouped by
    training horizon and a pixel's average daily demand (ADD) for a
    pixel size of $1~\text{km}^2$ and 60-minute time steps.
Each combination of pixel and test day counts as one case, and the total
    number of cases is denoted as $n$.
Clustering the individual results revealed that a pixel's ADD over the
    training horizon is the primary indicator of similarity and three to four
    clusters suffice to obtain cohesive clusters:
We labeled them "no", "low", "medium", and "high" demand pixels with
    increasing ADD, and present the average MASE per cluster.
The $n$ do not vary significantly across the training horizons, which confirms
    that the platform did not grow area-wise and is indeed in a steady-state.

\begin{center}
\captionof{table}{Top-3 models by training weeks and average demand
                  ($1~\text{km}^2$ pixel size, 60-minute time steps)}
\label{t:results}
\begin{tabular}{|c|c|*{12}{c|}}

\hline
\multirow{3}{*}{\rotatebox{90}{\thead{Training}}}
    & \multirow{3}{*}{\rotatebox{90}{\thead{Rank}}}
    & \multicolumn{3}{c|}{\thead{No Demand}}
    & \multicolumn{3}{c|}{\thead{Low Demand}}
    & \multicolumn{3}{c|}{\thead{Medium Demand}}
    & \multicolumn{3}{c|}{\thead{High Demand}} \\
~ & ~
    & \multicolumn{3}{c|}{(0 - 2.5)}
    & \multicolumn{3}{c|}{(2.5 - 10)}
    & \multicolumn{3}{c|}{(10 - 25)}
    & \multicolumn{3}{c|}{(25 - $\infty$)} \\
\cline{3-14}
~ & ~
    & Method & MASE & $n$
    & Method & MASE & $n$
    & Method & MASE & $n$
    & Method & MASE & $n$ \\

\hline \hline
\multirow{3}{*}{3} & 1
    & \textbf{\textit{trivial}}
        & 0.785 & \multirow{3}{*}{\rotatebox{90}{4586}}
    & \textbf{\textit{hsma}}
        & 0.819 & \multirow{3}{*}{\rotatebox{90}{2975}}
    & \textbf{\textit{hsma}}
        & 0.839 & \multirow{3}{*}{\rotatebox{90}{2743}}
    & \textbf{\textit{rtarima}}
        & 0.872 & \multirow{3}{*}{\rotatebox{90}{2018}} \\
~ & 2
    & \textit{hsma}    & 0.809 & ~
    & \textit{hses}    & 0.844 & ~
    & \textit{hses}    & 0.858 & ~
    & \textit{rtses}   & 0.873 & ~ \\
~ & 3
    & \textit{pnaive}  & 0.958 & ~
    & \textit{hets}    & 0.846 & ~
    & \textit{hets}    & 0.859 & ~
    & \textit{rtets}   & 0.877 & ~ \\

\hline
\multirow{3}{*}{4} & 1
    & \textbf{\textit{trivial}}
        & 0.770 & \multirow{3}{*}{\rotatebox{90}{4532}}
    & \textbf{\textit{hsma}}
        & 0.825 & \multirow{3}{*}{\rotatebox{90}{3033}}
    & \textbf{\textit{hsma}}
        & 0.837 & \multirow{3}{*}{\rotatebox{90}{2687}}
    & \textbf{\textit{vrfr}}
        & 0.855 & \multirow{3}{*}{\rotatebox{90}{2016}} \\
~ & 2
    & \textit{hsma}             & 0.788 & ~
    & \textit{hses}             & 0.848 & ~
    & \textit{hses}             & 0.850 & ~
    & \textbf{\textit{rtarima}} & 0.855 & ~ \\
~ & 3
    & \textit{pnaive}  & 0.917 & ~
    & \textit{hets}    & 0.851 & ~
    & \textit{hets}    & 0.854 & ~
    & \textit{rtses}   & 0.860 & ~ \\

\hline
\multirow{3}{*}{5} & 1
    & \textbf{\textit{trivial}}
        & 0.780 & \multirow{3}{*}{\rotatebox{90}{4527}}
    & \textbf{\textit{hsma}}
        & 0.841 & \multirow{3}{*}{\rotatebox{90}{3055}}
    & \textbf{\textit{hsma}}
        & 0.837 & \multirow{3}{*}{\rotatebox{90}{2662}}
    & \textbf{\textit{vrfr}}
        & 0.850 & \multirow{3}{*}{\rotatebox{90}{2019}} \\
~ & 2
    & \textit{hsma}             & 0.803 & ~
    & \textit{hses}             & 0.859 & ~
    & \textit{hets}             & 0.845 & ~
    & \textbf{\textit{rtarima}} & 0.852 & ~ \\
~ & 3
    & \textit{pnaive}  & 0.889 & ~
    & \textit{hets}    & 0.861 & ~
    & \textit{hses}    & 0.845 & ~
    & \textit{vsvr}    & 0.854 & ~ \\

\hline
\multirow{3}{*}{6} & 1
    & \textbf{\textit{trivial}}
        & 0.741 & \multirow{3}{*}{\rotatebox{90}{4470}}
    & \textbf{\textit{hsma}}
        & 0.847 & \multirow{3}{*}{\rotatebox{90}{3086}}
    & \textbf{\textit{hsma}}
        & 0.840 & \multirow{3}{*}{\rotatebox{90}{2625}}
    & \textbf{\textit{vrfr}}
        & 0.842 & \multirow{3}{*}{\rotatebox{90}{2025}} \\
~ & 2
    & \textit{hsma}             & 0.766 & ~
    & \textit{hses}             & 0.863 & ~
    & \textit{hets}             & 0.842 & ~
    & \textbf{\textit{hets}}    & 0.847 & ~ \\
~ & 3
    & \textit{pnaive}  & 0.837 & ~
    & \textit{hets}    & 0.865 & ~
    & \textit{hses}    & 0.848 & ~
    & \textit{vsvr}    & 0.848 & ~ \\

\hline
\multirow{3}{*}{7} & 1
    & \textbf{\textit{trivial}}
        & 0.730 & \multirow{3}{*}{\rotatebox{90}{4454}}
    & \textbf{\textit{hsma}}
        & 0.858 & \multirow{3}{*}{\rotatebox{90}{3132}}
    & \textbf{\textit{hets}}
        & 0.845 & \multirow{3}{*}{\rotatebox{90}{2597}}
    & \textbf{\textit{hets}}
        & 0.840 & \multirow{3}{*}{\rotatebox{90}{2007}} \\
~ & 2
    & \textit{hsma}          & 0.754 & ~
    & \textit{hses}          & 0.871 & ~
    & \textit{hsma}          & 0.847 & ~
    & \textbf{\textit{vrfr}} & 0.845 & ~ \\
~ & 3
    & \textit{pnaive}        & 0.813 & ~
    & \textit{hets}          & 0.872 & ~
    & \textbf{\textit{vsvr}} & 0.850 & ~
    & \textit{vsvr}          & 0.847 & ~ \\

\hline
\multirow{3}{*}{8} & 1
    & \textbf{\textit{trivial}}
        & 0.735 & \multirow{3}{*}{\rotatebox{90}{4402}}
    & \textbf{\textit{hsma}}
        & 0.867 & \multirow{3}{*}{\rotatebox{90}{3159}}
    & \textbf{\textit{hets}}
        & 0.846 & \multirow{3}{*}{\rotatebox{90}{2575}}
    & \textbf{\textit{hets}}
        & 0.836 & \multirow{3}{*}{\rotatebox{90}{2002}} \\
~ & 2
    & \textit{hsma}          & 0.758 & ~
    & \textit{hets}          & 0.877 & ~
    & \textbf{\textit{vsvr}} & 0.850 & ~
    & \textbf{\textit{vrfr}} & 0.842 & ~ \\
~ & 3
    & \textit{pnaive}  & 0.811 & ~
    & \textit{hses}    & 0.880 & ~
    & \textit{hsma}    & 0.851 & ~
    & \textit{vsvr}    & 0.849 & ~ \\

\hline
\end{tabular}
\end{center}
\

We use this table to answer \textbf{Q1} regarding the overall best methods
    under different ADDs.
All result tables in the main text report MASEs calculated with all time
    steps of a day.
In contrast, \ref{peak_results} shows the same tables with MASEs calculated
    with time steps within peak times only (i.e., lunch from 12 pm to 2 pm and
    dinner from 6 pm to 8 pm).
The differences lie mainly in the decimals of the individual MASE
    averages while the ranks of the forecasting methods do not change except
    in rare cases.
That shows that the presented accuracies are driven by the forecasting methods'
    accuracies at peak times.
Intuitively, they all correctly predict zero demand for non-peak times.

Unsurprisingly, the best model for pixels without demand (i.e.,
    $0 < \text{ADD} < 2.5$) is \textit{trivial}.
Whereas \textit{hsma} also adapts well, its performance is worse.
None of the more sophisticated models reaches a similar accuracy.
The intuition behind is that \textit{trivial} is the least distorted by the
    relatively large proportion of noise given the low-count nature of the
    time series.

For low demand (i.e., $2.5 < \text{ADD} < 10$), there is also a clear
    best-performing model, namely \textit{hsma}.
As the non-seasonal \textit{hses} reaches a similar accuracy as its
    potentially seasonal generalization, the \textit{hets}, we conclude that
    the seasonal pattern from weekdays is not yet strong enough to be
    recognized in low demand pixels.
So, in the absence of seasonality, models that only model a trend part are
    the least susceptible to the noise.

For medium demand (i.e., $10 < \text{ADD} < 25$) and training horizons up to
    six weeks, the best-performing models are the same as for low demand.
For longer horizons, \textit{hets} provides the highest accuracy.
Thus, to fit a seasonal pattern, longer training horizons are needed.
While \textit{vsvr} enters the top three, \textit{hets} has the edge as they
    neither require parameter tuning nor real-time data.

In summary, except for high demand, simple models trained on horizontal time
    series work best.
By contrast, high demand (i.e., $25 < \text{ADD} < \infty$) and less than
    six training weeks is the only situation where classical models trained on
    vertical time series work well.
Then, \textit{rtarima} outperforms their siblings from Sub-sections
    \ref{vert} and \ref{rt}.
We conjecture that intra-day auto-correlations as caused, for example, by
    weather, are the reason for that.
Intuitively, a certain amount of demand (i.e., a high enough signal-to-noise
    ratio) is required such that models with auto-correlations can see them
    through all the noise.
That idea is supported by \textit{vrfr} reaching a similar accuracy under
    high demand as their tree-structure allows them to fit auto-correlations.
As both \textit{rtarima} and \textit{vrfr} incorporate recent demand,
    real-time information can indeed improve accuracy.
However, once models are trained on longer horizons, \textit{hets} is more
    accurate than \textit{vrfr}.
Thus, to answer \textbf{Q4}, we conclude that real-time information only
    improves accuracy if three or four weeks of training material are
    available.

In addition to looking at the results in tables covering the entire one-year
    horizon, we also created sub-analyses on the distinct seasons spring,
    summer (incl. the long holiday season in France), and fall.
Yet, none of the results portrayed in this and the subsequent sections change
    is significant ways.
We conjecture that there could be differences if the overall demand of the UDP
    increased to a scale beyond the one this case study covers and leave that
    up to a follow-up study with a bigger UDP.
