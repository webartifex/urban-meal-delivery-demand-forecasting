\subsection{Impact of the Training Horizon}
\label{training}

Whereas it is reasonable to assume that forecasts become more accurate as the
    training horizon expands, our study reveals some interesting findings.
First, without demand, \textit{trivial} indeed performs better with more
    training material, but improved pattern recognition cannot be the cause
    here.
Instead, we argue that the reason for this is that the longer there has been
    no steady demand, the higher the chance that this will not change soon.
Further, if we focus on shorter training horizons, the sample will necessarily
    contain cases where a pixel is initiated after a popular-to-be restaurant
    joined the platform:
Demand grows fast making \textit{trivial} less accurate, and the pixel moves
    to another cluster soon.

Second, with low demand, the best-performing \textit{hsma} becomes less
    accurate with more training material.
While one could argue that this is due to \textit{hsma} not fitting a trend,
    the less accurate \textit{hses} and \textit{hets} do fit a trend.
Instead, we argue that any low-demand time series naturally exhibits a high
    noise-to-signal ratio, and \textit{hsma} is the least susceptible to
    noise.
Then, to counter the missing trend term, the training horizon must be shorter.

With medium demand, a similar argument can be made; however, the
    signal already becomes more apparent favoring \textit{hets} with more
    training data.

Lastly, with high demand, the signal becomes so clear that more sophisticated
    models can exploit longer training horizons.
