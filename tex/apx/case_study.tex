\section{Raw Order Data in the Case Study}
\label{dataset}

The raw data for the empirical study in Section \ref{stu} was provided by a
    meal delivery platform operating in five cities in France in 2016.
The platform received a total of 686,385 orders distributed as follows:

\
\begin{center}
\begin{tabular}{llr}
    \hline
	\thead{City} & \thead{Launch Day} & \thead{Orders} \\
	\hline
    Bordeaux & July 18 & 64,012 \\
    Lille & October 30 & 14,362 \\
    Lyon & February 21 & 214,635 \\
    Nantes & October 31 & 12,900 \\
    Paris & March 7 & 380,476 \\
\end{tabular}
\end{center}
\

The part of the database relevant for forecasting can be thought of as one
    table per city, where each row represents one order and consists of the
    following groups of columns:
\begin{enumerate}
\item \textbf{Restaurant Data}
    \begin{enumerate}
    \item unique ID and name
    \item pickup location as latitude-longitude pair
    \end{enumerate}
\item \textbf{Customer Data}
    \begin{enumerate}
    \item unique ID, name, and phone number
    \item delivery location as latitude-longitude pair (mostly physical
	      addresses but also public spots)
	\end{enumerate}
\item \textbf{Timestamps}
    \begin{enumerate}
    \item placement via the smartphone app
    \item fulfillment workflow (pickup, delivery, cancellation, re-deliveries)
    \end{enumerate}
\item \textbf{Courier Data}
    \begin{enumerate}
    \item unique ID, name, and phone number
    \item shift data (begin, breaks, end)
    \item average speed
    \end{enumerate}
\item \textbf{Order Details}
    \begin{enumerate}
    \item meals and drinks
    \item prices and discounts granted
    \end{enumerate}
\end{enumerate}
