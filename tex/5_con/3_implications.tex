\subsection{Managerial Implications}
\label{implications}

Even though zeitgeist claims that having more data is always better, our study
    shows this is not the case here:
First, under certain circumstances, accuracy may go up with shorter training
    horizons.
Second, none of the external data sources improves the accuracies.
Somewhat surprisingly, despite ML-based methods` popularity in both business
    and academia in recent years, we must conclude that classical forecasting
    methods suffice to reach the best accuracy in our study.
There is one case where ML-based methods are competitive in our case study:
    In a high demand pixel (defined as more than 25 orders per day on average),
    if only about four to six weeks of past data is available,
    the \textit{vrfr} model outperformed the classical ones.
So, we recommend trying out ML-based methods in such scenarios.
In addition, with the \textit{hsma} and \textit{hets} models being the overall
    winners, incorporating real-time data is not beneficial, in particular,
    with more than six weeks of training material available.
Lastly, with just \textit{hets}, that exhibits an accuracy comparable to
    \textit{hsma} for low and medium demand, our industry partner can likely
    schedule its shifts on an hourly basis one week in advance.

This study gives rise to the following managerial implications.
First, UDPs can implement readily available forecasting algorithms with limited
    effort.
This, however, requires purposeful data collection and preparation by those
    companies, which, according to our study, is at least equally important as
    the selection of the forecasting algorithm, as becomes clear from 
    investigating the impact of the length of the training horizon.
Second, the benefits of moving from manual forecasting to automated forecasting
    include being able to pursue a predictive routing strategy and
    demand-adjusted shift scheduling.
At the time the case study data was collected, our industry partner did not
    conduct any forecasting; the only forecasting-related activities were the
    shift managers scheduling the shifts one week in advance manually in 
    spreadsheets.
Thus, selecting the right forecasting algorithm according to the framework
    proposed in this study becomes a prerequisite to the much needed
    operational improvements UDPs need to achieve in their quest for
    profitability.
In general, many UDPs launched in recent years are venture capital backed
    start-up companies that almost by definition do not have a strong
    grounding in operational excellence, and publications such as the ones by
    Uber are the exception rather than the rule.
Our paper shows that forecasting the next couple of hours can already be
    implemented within the first year of a UDP's operations.
Even if such forecasts could not be exploited by predictive routing (e.g., due
    to prolonged waiting times at restaurants), they would help monitoring the
    operations for exceptional events.
Additionally, the shift planning may be automated saving as much as one shift
    manager per city.
We emphasize that for the most part, our proposed forecasting system
    is calibrated automatically and no manual work by a data scientist is required.
The only two parameters where assumptions need to be made are the pixel size
    and the time step.
The results in our empirical study suggest
    that a pixel size of $1~\text{km}^2$ and a time step of one hour are ideal,
    which results in the optimal trade-off
    between signal strength and spatial-temporal resolution.
Future research may explore adaptive grid-sizing depending on, for instance, demand density.