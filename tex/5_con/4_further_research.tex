\subsection{Further Research}
\label{further_research}

Sub-sections \ref{overall_results} and \ref{fams} present the models' average
    performance.
We did not research what is the best model in a given pixel on a given day.
To answer this, a study finding an optimal number of outer validation days is
    neccessary.
With the varying effect of the training horizon, this model selection is a
    two-dimensional grid search that is prone to overfitting due to the high
    noise in low count data.
Except heuristics relating the ADD to the training horizon, we cannot say
    anything about that based on our study.
\cite{lemke2010} and \cite{wang2009} show how, for example, a time series'
    characteristics may be used to select models.
Thus, we suggest conducting more detailed analyses on how to incorporate model
    selection into our proposed forecasting system.

Future research should also integrate our forecasting system into a predictive
    routing application and evaluate its business impact.
This embeds our research into the vast literature on the VRP.
Initially introduced by \cite{dantzig1959}, \gls{vrp}s are concerned with
    finding optimal routes serving customers.
We refer to \cite{toth2014} for a comprehensive overview.
The two variants relevant for the UDP case are the dynamic VRP and
    the pickup and delivery problem (\gls{pdp}).
A VRP is dynamic if the data to solve a problem only becomes available
    as the operations are underway.
\cite{thomas2010}, \cite{pillac2013}, and \cite{psaraftis2016} describe how
    technological advances, in particular, mobile technologies, have led to a
    renewed interest in research on dynamic VRPs, and
    \cite{berbeglia2010} provide a general overview.
\cite{ichoua2006} and \cite{ferrucci2013} provide solution methods for
    simulation studies where they assume stochastic customer demand based on
    historical distributions.
In both studies, dummy demand nodes are inserted into the VRP instance.
Forecasts by our system extend this idea naturally as dummy nodes could be
    derived from point forecasts as well.
The concrete case of a meal delivering UDP is contained in a recent
    literature stream started by \cite{ulmer2017} and extended by
    \cite{reyes2018} and \cite{yildiz2018}: They coin the term meal delivery
    routing problem (\gls{mdrp}).
The MDRP is a special case of the dynamic PDP where the defining
    characteristic is that once a vehicle is scheduled, a modification of the
    route is inadmissible.
