\begin{frontmatter}

\journal{Transportation Research Part E}
\title{Real-time Demand Forecasting for an Urban Delivery Platform}

\author[WHU]{Alexander Hess\fnref{emails}}
\author[WHU]{Stefan Spinler\fnref{emails}}
\author[MIT]{Matthias Winkenbach\fnref{emails}}
\address[WHU]{
WHU - Otto Beisheim School of Management,
Burgplatz 2, 56179 Vallendar, Germany
}
\address[MIT]{
Massachusetts Institute of Technology,
77 Massachusetts Avenue, Cambridge, MA 02139, United States
}
\fntext[email]{
Emails:
    alexander.hess@whu.edu,
    stefan.spinler@whu.edu,
    mwinkenb@mit.edu
}

\begin{abstract}
Meal delivery platforms like Uber Eats shape the landscape in cities around the world.
This paper addresses forecasting demand into the short-term future.
We propose an approach incorporating
    both classical forecasting
    and machine learning methods.
Model evaluation and selection is adapted to demand typical for such a platform
    (i.e., intermittent with a double-seasonal pattern).
The results of an empirical study with a European meal delivery service show
    that machine learning models become competitive
    once the average daily demand passes a threshold.
As a main contribution, the paper explains
    how a forecasting system must be set up
    to enable predictive routing.
\end{abstract}

\begin{keyword}
demand forecasting \sep
intermittent demand \sep
machine learning \sep
urban delivery platform
\end{keyword}

\end{frontmatter}