\begin{frontmatter}

\journal{Transportation Research Part E}
\title{Real-time Demand Forecasting for an Urban Delivery Platform}

\author[WHU]{Alexander Hess\fnref{emails}}
\author[WHU]{Stefan Spinler\fnref{emails}}
\author[MIT]{Matthias Winkenbach\fnref{emails}}
\address[WHU]{
WHU - Otto Beisheim School of Management,
Burgplatz 2, 56179 Vallendar, Germany
}
\address[MIT]{
Massachusetts Institute of Technology,
77 Massachusetts Avenue, Cambridge, MA 02139, United States
}
\fntext[email]{
Emails:
    alexander.hess@whu.edu,
    stefan.spinler@whu.edu,
    mwinkenb@mit.edu
}

\begin{abstract}
Meal delivery platforms like Uber Eats shape the landscape in cities around the world.
This paper addresses forecasting demand on a grid into the short-term future,
    enabling, for example, predictive routing applications.
We propose an approach incorporating
    both classical forecasting and machine learning methods
    and adapt model evaluation and selection to typical demand:
        intermittent with a double-seasonal pattern.
An empirical study shows that
    an exponential smoothing based method trained on past demand data alone
        achieves optimal accuracy,
    if at least two months are on record.
With only a shorter demand history,
    machine learning methods and real-time data may improve prediction.
\end{abstract}

\begin{keyword}
demand forecasting \sep
intermittent demand \sep
machine learning \sep
urban delivery platform
\end{keyword}

\end{frontmatter}