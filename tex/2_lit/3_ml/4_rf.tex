\subsubsection{Random Forest Regression}
\label{rf}

\cite{breiman1984} introduce the classification and regression tree
    (\gls{cart}) model that is built around the idea that a single binary
    decision tree maps learned combinations of intervals of the feature
    columns to a label.
Thus, each sample in the training set is associated with one leaf node that
    is reached by following the tree from its root and branching along the
    arcs according to some learned splitting rule per intermediate node that
    compares the sample's realization for the feature specified by the rule to
    the learned decision rule.
While such models are computationally fast and offer a high degree of
    interpretability, they tend to overfit strongly to the training set as
    the splitting rules are not limited to any functional form (e.g., linear)
    in the relationship between the features and the labels.
In the regression case, it is common to maximize the variance reduction $I_V$
    from a parent node $N$ to its two children, $C1$ and $C2$, as the
    splitting rule.
\cite{breiman1984} formulate this as follows:
$$
I_V(N)
=
\frac{1}{|S_N|^2} \sum_{i \in S_N} \sum_{j \in S_N}
    \frac{1}{2} (y_i - y_j)^2
- \left(
    \frac{1}{|S_{C1}|^2} \sum_{i \in S_{C1}} \sum_{j \in S_{C1}}
        \frac{1}{2} (y_i - y_j)^2
    +
    \frac{1}{|S_{C2}|^2} \sum_{i \in S_{C2}} \sum_{j \in S_{C2}}
        \frac{1}{2} (y_i - y_j)^2
\right)
$$
$S_N$, $S_{C1}$, and $S_{C2}$ are the index sets of the samples in $N$, $C1$,
    and $C2$. 

\cite{ho1998} and then \cite{breiman2001} generalize this method by combining
    many CART models into one forest of trees where every single tree is
    a randomized variant of the others.
Randomization is achieved at two steps in the training process:
First, each tree receives a distinct training set resampled with replacement
    from the original training set, an idea also called bootstrap
    aggregation.
Second, at each node a random subset of the features is used to grow the tree.
Trees can be fitted in parallel speeding up the training significantly.
For prediction at the tree level, the average of all the samples at a
    particular leaf node is used.
Then, the individual values are combined into one value by averaging again
    across the trees.
Due to the randomization, the trees are decorrelated offsetting the
    overfitting.
Another measure to counter overfitting is pruning the tree, either by
    specifying the maximum depth of a tree or the minimum number of samples
    at leaf nodes.

The forecaster must tune the structure of the forest.
Parameters include the number of trees in the forest, the size of the random
    subset of features, and the pruning criteria.
The parameters are optimized via grid search: We train many models with
    parameters chosen from a pre-defined list of values and select the best
    one by CV.
RFs are a convenient ML method for any dataset as decision trees do not
    make any assumptions about the relationship between features and labels.
\cite{herrera2010} use RFs to predict the hourly demand for water in an urban
    context, a similar application as the one in this paper, and find that RFs
    work well with time series type of data.
