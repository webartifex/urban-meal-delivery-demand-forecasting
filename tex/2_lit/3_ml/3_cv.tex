\subsubsection{Cross-Validation}
\label{cv}

Because ML models are trained by minimizing a loss function $L$, the
    resulting value of $L$ underestimates the true error we see when
    predicting into the actual future by design.
To counter that, one popular and model-agnostic approach is cross-validation
    (\gls{cv}), as summarized, for example, by \cite{hastie2013}.
CV is a resampling technique, which ranomdly splits the samples into a
    training and a test set.
Trained on the former, an ML model makes forecasts on the latter.
Then, the value of $L$ calculated only on the test set gives a realistic and
    unbiased estimate of the true forecasting error, and may be used for one
    of two distinct aspects:
First, it assesses the quality of a fit and provides an idea as to how the
    model would perform in production when predicting into the actual future.
Second, the errors of models of either different methods or the same method
    with different parameters may be compared with each other to select the
    best model.
In order to first select the best model and then assess its quality, one must
    apply two chained CVs:
The samples are divided into training, validation, and test sets, and all
    models are trained on the training set and compared on the validation set.
Then, the winner is retrained on the union of the training and validation
    sets and assessed on the test set.

Regarding the splitting, there are various approaches, and we choose the
    so-called $k$-fold CV, where the samples are randomly divided into $k$
    folds of the same size.
Each fold is used as a test set once and the remaining $k-1$ folds become
    the corresponding training set.
The resulting $k$ error measures are averaged.
A $k$-fold CV with $k=5$ or $k=10$ is a compromise between the two extreme
    cases of having only one split and the so-called leave-one-out CV
    where $k = m$: Computation is still relatively fast and each sample is
    part of several training sets maximizing the learning from the data.
We adapt the $k$-fold CV to the ordinal stucture in $\mat{X}$ and $\vec{y}$ in
    Sub-section \ref{unified_cv}.
