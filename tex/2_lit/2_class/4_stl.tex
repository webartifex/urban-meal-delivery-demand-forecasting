\subsubsection{Seasonal and Trend Decomposition using Loess}
\label{stl}

A time series $y_t$ may exhibit different types of patterns; to fully capture
    each of them, the series must be decomposed.
Then, each component is forecast with a distinct model.
Most commonly, the components are the trend $t_t$, seasonality $s_t$, and
    remainder $r_t$.
They are themselves time series, where only $s_t$ exhibits a periodicity $k$.
A decomposition may be additive (i.e., $y_t = s_t + t_t + r_t$) or
    multiplicative (i.e., $y_t = s_t * t_t * r_t$); the former assumes that
    the effect of the seasonal component is independent of the overall level
    of $y_t$ and vice versa.
The seasonal component is centered around $0$ in both cases such that its
    removal does not affect the level of $y_t$.
Often, it is sufficient to only seasonally adjust the time series, and model
    the trend and remainder together, for example, as $a_t = y_t - s_t$ in the
    additive case.

Early approaches employed moving averages (cf., Sub-section \ref{ets}) to
    calculate a trend component, and, after removing that from $y_t$, averaged
    all observations of the same seasonal lag to obtain the seasonal
    component.
The downsides of this are the subjectivity in choosing the window lengths for
    the moving average and the seasonal averaging, the incapability of the
    seasonal component to vary its amplitude over time, and the missing
    handling of outliers.

The X11 method developed at the U.S. Census Bureau and described in detail by
    \cite{dagum2016} overcomes these disadvantages.
However, due to its background in economics, it is designed primarily for
    quarterly or monthly data, and the change in amplitude over time cannot be
    controlled.
Variants of this method are the SEATS decomposition by the Bank of Spain and
    the newer X13-SEATS-ARIMA method by the U.S. Census Bureau.
Their main advantages stem from the fact that the models calibrate themselves
    according to statistical criteria without manual work for a statistician
    and that the fitting process is robust to outliers.

\cite{cleveland1990} introduce a seasonal and trend decomposition using a
    repeated locally weighted regression - the so-called Loess procedure - to
    smoothen the trend and seasonal components, which can be viewed as a
    generalization of the methods above and is denoted by the acronym STL.
In contrast to the X11, X13, and SEATS methods, the STL supports seasonalities
    of any lag $k$ that must, however, be determined with additional
    statistical tests or set with out-of-band knowledge by the forecaster
    (e.g., hourly demand data implies $k = 24 * 7 = 168$ assuming customer
    behavior differs on each day of the week).
Moreover, the seasonal component's rate of change, represented by the $ns$
    parameter and explained in detail with Figure \ref{f:stl} in Section
    \ref{decomp}, must be set by the forecaster as well, while the trend's
    smoothness may be controlled via setting a non-default window size.
Outliers are handled by assignment to the remainder such that they do not
    affect the trend and seasonal components.
In particular, the manual input needed to calibrate the STL explains why only
    the X11, X13, and SEATS methods are widely used by practitioners.
However, the widespread adoption of concepts like cross-validation (cf.,
    Sub-section \ref{cv}) in recent years enables the usage of an automated
    grid search to optimize the parameters.
The STL's usage within a grid search is facilitated even further by its being
    computationally cheaper than the other methods discussed.
